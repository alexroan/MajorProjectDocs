\chapter{Evaluation}

Examiners expect to find in your dissertation a section addressing such questions as:

\begin{itemize}
   \item Were the requirements correctly identified? 
   \item Were the design decisions correct?
   \item Could a more suitable set of tools have been chosen?
   \item How well did the software meet the needs of those who were expecting to use it?
   \item How well were any other project aims achieved?
   \item If you were starting again, what would you do differently?
\end{itemize}

Such material is regarded as an important part of the dissertation; it should demonstrate that you are capable not only of carrying out a piece of work but also of thinking critically about how you did it and how you might have done it better. This is seen as an important part of an honours degree. 

There will be good things and room for improvement with any project. As you write this section, identify and discuss the parts of the work that went well and also consider ways in which the work could be improved. 

Review the discussion on the Evaluation section from the lectures. A recording is available on Blackboard. 

\section{Original Goals}
The original goals set out in the project specification 

\section{Accomplishments}
I believe that the original goals that were set out from the beginning were accomplished to a good standard. I also believe however, that the project deserves a more extensive result. The purpose of the website was always to provide a solution to a wide range of people, and to provide a wide range of services so as to encourage a large user base. Unfortunately, I think the intended size of the project I had planned from the beginning was perhaps larger than was realistically possible which resulted in the specifications being as small in number as they are. 

The site performs the type of search that I emphasised. Enabling journeys to be found which cover points along the routes, and then changing the routes depending on pick up points and drop off points.

\section{Possible Improvements}


\section{Future Development}
The future development of this project would involve the following features:

\begin{itemize}
\item The improvement of the website interface. Currently, the website does perform the tasks it was set out to provide to the user, but could be improved by analysing the current state of interaction between human and site with questionnaires.
\item 
\end{itemize}

\section{Design Choices}
\subsection{Database Controller}
I think the choices I made regarding design of the project could have been better. In hindsight, the design of the database controllers and website could have been much more structured in the form of a standard design pattern. Although it is a tweaked version of the Model-View-Controller pattern, I think it came to be as a result of indecision regarding the design of the site. If more research was done towards Representational State Transfer (RESTful) services I would have probably designed the project to follow that instead. This is due to RESTful services being so easily expandable, and would probably have allowed me to more easily add extra features to the site once the structure of the project had been developed.

\subsection{Website}
The website developed was completely procedural, and made use of the object oriented controller classes when required. Although this allowed me to develop a completely customised website to my own specifications, it did mean that each file that I produced had to be written from scratch. There are obvious advantages to this, such as being able to develop customised pages for whatever function the page is intended to provide, but the time it took to create them and the amount of separate files that needed to be created was a potential hindrance to the project.

\subsection{Database}
I feel the design of the database was as simple as could have been possibly made. It enables the site to store as much important information as possible whilst reducing the complexity of the queries that need to be made to extract related information from multiple tables. 

\section{Approach}
The approach to the design of this project could have been performed to a better standard. Initially, not sure whether I wanted to develop using an agile methodology or a more structured one like the Waterfall method meant that I deviated from any methodology slightly when developing prototypes to test some features without any planning. Although I did gain some valuable information from this, I feel that it may have wasted time that could have been better spent planning, researching and designing the system at a higher level.

In hindsight, I do believe that the Waterfall method was the correct method to develop this project. Although I did not follow its steps strictly, I believe that it enabled me to make architectural decisions regarding the database which would not have developed with a sensible structure had I used an agile approach such as Feature Driven Development. 
