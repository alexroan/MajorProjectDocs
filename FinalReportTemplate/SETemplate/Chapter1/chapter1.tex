\chapter{Background \& Objectives}

This section should discuss your preparation for the project, including background reading, your analysis of the problem and the process or method you have followed to help structure your work.  It is likely that you will reuse part of your outline project specification, but at this point in the project you should have more to talk about. 

\textbf{Note}: 

\begin{itemize}
   \item All of the sections and text in this example are for illustration purposes. The main Chapters are a good starting point, but the content and actual sections that you include are likely to be different.
   
   \item Look at the document on the Structure of the Final Report for additional guidance. 
   
\end {itemize}

\section{Background}
What was your background preparation for the project? What similar systems did you assess? What was your motivation and interest in this project? 
\subsection{Overview}
\subsection{Example Use Case}
	Upon opening the home page, the user will be confronted with  the option of either registering, or logging in using existing details. Either action done successfully will log the user in.
	
	Once logged in, the user is taken to their home page, which displays predicted journeys that the site thinks the user might be interested in using their preferences. From this page, they can access all features within the site, which include:
	\begin{itemize}
		\item 'My Activity' - Information about upcoming rides and hitches related to the user. The ability to accept or decline hitch requests for their journeys.
		\item 'Messages' - Messages to and from other users.
		\item 'My Profile' - Personal details about the user, available for editing.
		\item 'Find a Journey' - A search page used to search for journeys from one location to another.
		\item 'Post Journey' - The page used to post new journeys that the user will be partaking in.		
	\end{itemize}
	
	When a user posts a journey from location A to location B using the 'Post Journey' page, it will display a map route which needs to be accepted by the user. Once this is accepted, the journey will be entered into the database, where it will be returned in other users searches if the parameters match the details of the journey.
	
	If another user requests to hitch a ride on the journey, the driver will be prompted in the 'My Activity' page, and will have the option of accepting or declining the hitch request. 
	
	The hitch request could be from location A to location B, but it could also be from location A to a new location C. Similarly it could be from location C to location B, or even location C to location D. It is up to the driver to accept or decline the hitch request, depending on how it changes the route that the driver takes.
	
	If the driver accepts this new hitcher, the route will be altered if necessary to include the new pickup / drop off points and saved into the database with one of the spare spaces now filled. 
\subsection{Existing Services}
	There are existing websites that offer the same sort of carpooling service. The leading service is Carpooling.com \cite{carpooling_com}. This site offers users the opportunity to both post lifts and request lifts from other users, much like my service. Currently, this is the largest online carpooling service in the UK.
	
\subsection{How this Compares}
	When I began this project, the existing carpooling website Carpooling.com required that users outline the towns and cities on the route at which they were taking for their journey. Now however, on top of this specification of place names being driven through, they calculate some of the specific places that are being passed. This is what I have designed my project to do.

\section{Analysis}
Taking into account the problem and what you learned from the background work, what was your analysis of the problem? How did your analysis help to decompose the problem into the main tasks that you would undertake? Were there alternative approaches? Why did you choose one approach compared to the alternatives? 

There should be a clear statement of the objectives of the work, which you will evaluate at the end of the work. 

In most cases, the agreed objectives or requirements will be the result of a compromise between what would ideally have been produced and what was felt to be possible in the time available. A discussion of the process of arriving at the final list is usually appropriate.

\subsection{Original Goals}
	The original goals of the project were to produce an online service that enables people to share journeys with people who are looking to travel to and from similar locations or locations along the route of the driver. 
	
\subsection{Requirements}
\subsubsection{Interface Requirements}
\subsubsection{Functional Requirements}
\subsubsection{Database Requirements}
\subsubsection{Software Requirements}
\subsubsection{Performance Requirements}

\section{Process}
You need to describe briefly the life cycle model or research method that you used. You do not need to write about all of the different process models that you are aware of. Focus on the process model that you have used. It is possible that you needed to adapt an existing process model to suit your project; clearly identify what you used and how you adapted it for your needs.

\subsection{Overview}
\subsection{Methodology}
\subsection{Planning}
\subsection{Research}
