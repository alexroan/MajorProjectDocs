\chapter{Background \& Objectives}
\section{Background}
\subsection{Overview}
As a student studying in a university away from home, I have often needed to travel back home during the holidays or on weekends. My motivation for this project comes from the requirement that many people have to travel frequently from one place to another. 

I have always driven to Cardiff from Aberystwyth and back during the holidays in a car with 4 spare seats. If there was some way I could offer those spare seats to other students or people travelling in the same or similar direction the cost of fuel could be split, meaning a cheaper mode of transport.

With the cost of public transport soaring ever higher\cite{gov_public_transport_doc}, travelling around the country is becoming a more expensive task year after year. Not only are bus and train prices increasing, but petrol prices have sky rocketed in the last century\cite{guardian_fuel_prices} as well. This website aims to bring people travelling in similar directions together, to save on the cost of fuel instead of having to pay a full train fair or travelling alone. 

A one way train ticket between Aberystwyth and Cardiff costs around \pounds 54.80\cite{trainline_aber_cardiff_price}, a journey that in a small car can cost as little as \pounds 20 in fuel. If that journey is shared between a driver and a passenger, who would otherwise have no option but to take the train, they would each only need to pay \pounds 10 for the journey. The driver would save \pounds 10 on their journey, and the passenger would save \pounds 44.80.


\subsection{Example Use Case}
	Upon opening the home page, the user will be confronted with the option of either registering, or logging in using existing details. Either action done successfully will log the user in.
	
	Once logged in, the user is taken to their home page, which displays predicted journeys that the site thinks the user might be interested in using their preferences. From this page, they can access all features within the site, which include:
	\begin{itemize}
		\item 'My Activity' - Information about upcoming rides and hitches related to the user. The option to accept or decline hitch requests for their journeys.
		\item 'Messages' - Messages to and from other users.
		\item 'My Profile' - Personal details about the user, available for editing.
		\item 'Find a Journey' - A search page used to search for journeys from one location to another.
		\item 'Post Journey' - The page used to post new journeys that the user will be partaking in.		
	\end{itemize}
	
	When a user posts a journey from location A to location B using the 'Post Journey' page, it will display a map route which needs to be accepted by the user. Once this is accepted, the journey will be entered into the database, where it will be returned in other user's searches if the parameters match the details of the journey. These details could be two locations along the journey that are not the origin and destination, but somewhere in between.
	
	If another user requests to hitch a ride on the journey, the driver will be prompted in the 'My Activity' page, and will have the option of accepting or declining the hitch request. 
	
	The hitch request could be from the origin to the destination, but it could also be from origin to a new location along the route of the original journey. Similarly, it could be from a location along the route of the journey to the journey destination, or even from one location along the route somewhere to another. It is up to the driver to accept or decline the hitch request, depending on how it changes the route that the driver takes.
	
	If the driver accepts this new hitcher, the route will be altered if necessary to include the new pickup / drop off points and saved into the database with one of the spare spaces now filled.
	 
\subsection{Existing Services}
	There are existing websites that offer the same sort of carpooling service. The leading services are Carpooling.com \cite{carpooling_com} and BlaBlaCar.com \cite{blabla}. These sites provide a platform for people to offer rides to other people needing to travel in similar directions.
	
	\begin{description}
	\item[Carpooling.com] This is a worldwide service which offers registered users the chance to share rides to other users.
	\item[Blablacar.com] BlaBlaCar is the largest car sharing network in Europe and rates users on their experience usign the site.
	\end{description}
	

\section{Analysis}

\subsection{Original Goals}
	The original goals of the project were to produce an online service that enables people to share journeys with people who are looking to travel to and from similar locations or locations along the route of the driver, this was the key marker as to the success of the project. The service must recognise when two locations in a search lie along the route of an existing journey with different origin and destination, not just recognise the searched locations as the beginning and end of a journey. In other words, the service must account for picking people up and dropping people off at any point in the journey, should the driver accept the hitcher.

The most important goals of the project are as follows:
	
\subsection{Requirements}
\begin{itemize}
\item Register as a new user to the site
\item Log in with existing details
\item Edit profile details
\item Post a journey that the user will be driving from location A to location B
\item Search for journeys between two locations. The search must return journeys in which the searched location appear along some points on the journeys
\item Request to hitch a journey as a passenger
\item Accept or decline a hitcher as the driver
\item Send messages to other users on the site
\end{itemize}

Ideally, provided a more extensive time period, the requirements would involve more in-depth services. These include the like of integration with social networking sites and the introduction of interoperability such as OAuth\cite{oauth} to extract data from existing user account to different services. Also, a service like this would benefit from being accessible away from a traditional desktop computer, with a mobile application being developed. Unfortunately, time restrictions and priority on the main purpose of the project meant that these features were omitted from the initial requirements stage.

\section{Process}
\subsection{Overview}
Planning and research was performed before the implementation of this project. Before the development could begin, the method in which to develop needed to be established. There were a few methodologies considered, detailed in the Methodology section of this chapter.

Research into the types of technologies I needed to use was also required, prompting reading into APIs, code libraries and programming languages. These are detailed in the Planning and Research sections of this chapter.

\subsection{Methodology}
Agile methodologies were considered for the development of this project. The decision lied mainly between choosing Feature Driven Development (FDD) along with Test Driven Development (TDD) as the method of coding the site, or using the more structured Waterfall method of development. There were pros and cons to both of these methods. 

Feature Driven Development would allow a very flexible environment in which to develop each feature iteratively, without much thought into future architecture. Although this would account for any future changes in specification or problems encountered during the implementation stage, I did not feel as though it would suit the development of the site. This is due to the lack of architectural planning involved in agile methodologies such as FDD. I feel that a project as reliant on its data storage method such as this requires more thought towards the design of such data storage methods. A tweaked version of FDD was still an option, allowing for a plan and design of the database, with the incremental development of each of the features implemented over time using the original database design which would be developed first. This however, defeats the object of agile development because it would not really allow for any future changes due to the architecture being set in stone from the beginning.

The Waterfall methodology requires that specifications and design are compiled well before implementation begins, and I feel that this suited the project more. The architecture of the database, and program controlling data flow to and from it, could be planned in advance and implemented within a short period of time. Although this did not account much for future changes, the architecture could be discussed in detail and designed with planning rather than be developed on the fly, resulting in a more normalised relational database.

\subsection{Research}
The vast majority of this service depends on its ability to recognise, locate and route directions from an origin to a destination. It was clear from the beginning that either a routing algorithm needed to be developed using real map data, or an existing or multiple existing APIs needed to be utilized to produce routing data that could be saved locally to the site's database for processing. 

Google Directions API\cite{google_directions_api} was found to be a valuable asset in the projects development. Using it enables routes to be created from an origin, destination and way points. The result that gets returned can be specified by the parameters parsed to it, for example: JSON, XML, etc. It details the global position of the origin and destination names parsed to it, and the global position of each step taken along the route. This proved to be perfect for the types of requests that this project needs to make. 

Another valuable API that was researched was the Google Geocoding API\cite{google_geocoding_api}. This API returns the geographical location in latitude and longitude of any place name given as a parameter, useful when searching using the geographical location of places along the route of journeys.

\subsection{Planning}
The planning that was carried out followed requirements specified by the Waterfall development methodology. Initially, a project specification document was developed to outline the requirements of the project. Once they had been established, the design of the project began to develop, resulting in some basic UML diagrams describing the database structure and website. The Waterfall process deviated slightly from its normally strict processes here where a few prototypes were developed to test out some of the design theories compiled during the design stage. Any issues or flaws in the design were corrected at this stage, enabling for the smoothest implementation possible.


