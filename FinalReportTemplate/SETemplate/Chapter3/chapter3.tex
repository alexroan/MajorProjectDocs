\chapter{Implementation}

The implementation should look at any issues you encountered as you tried to implement your design. During the work, you might have found that elements of your design were unnecessary or overly complex; perhaps third party libraries were available that simplified some of the functions that you intended to implement. If things were easier in some areas, then how did you adapt your project to take account of your findings?

It is more likely that things were more complex than you first thought. In particular, were there any problems or difficulties that you found during implementation that you had to address? Did such problems simply delay you or were they more significant? 

You can conclude this section by reviewing the end of the implementation stage against the planned requirements. 

\section{Overview}
\section{PHP}
	A few languages were considered for this project, these include Ruby, Python and Perl. Perl was the first to be discarded. This was mainly due to its performance and usability when used in an Object Oriented fashion. Ruby with Ruby of Rails provides a very stable platform to develop upon, but does not quite have the flexibility that I intended to yield in the development. The final decision came between Python and PHP. Python is a clean language with very good performance and is easy to use. However, the flexibility that PHP provides when developing web pages and its good Object Oriented capabilities meant that I sided with it even though its performance may not be as good as Python.
\section{Google Directions API}
	-talk about long motorway journeys and how steps can be very far apart
	\section{Database}
	\section{Database API}
	-procedural v OO
	\section{Website}
	-communication with API.
