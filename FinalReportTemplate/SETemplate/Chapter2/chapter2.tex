%\addcontentsline{toc}{chapter}{Development Process}
\chapter{Design}

You should concentrate on the more important aspects of the design. It is essential that an overview is presented before going into detail. As well as describing the design adopted it must also explain what other designs were considered and why they were rejected.

The design should describe what you expected to do, and might also explain areas that you had to revise after some investigation.

Typically, for an object-oriented design, the discussion will focus on the choice of objects and classes and the allocation of methods to classes. The use made of reusable components should be described and their source referenced. Particularly important decisions concerning data structures usually affect the architecture of a system and so should be described here.

How much material you include on detailed design and implementation will depend very much on the nature of the project. It should not be padded out. Think about the significant aspects of your system. For example, describe the design of the user interface if it is a critical aspect of your system, or provide detail about methods and data structures that are not trivial. Do not spend time on long lists of trivial items and repetitive descriptions. If in doubt about what is appropriate, speak to your supervisor.
 
You should also identify any support tools that you used. You should discuss your choice of implementation tools - programming language, compilers, database management system, program development environment, etc.

Some example sub-sections may be as follows, but the specific sections are for you to define. 

\section{Overview}
\section{Technologies}
	\subsection{PHP}
		A few languages were considered for this project, these include Ruby, Python and Perl. Perl was the first to be discarded. This was mainly due to its performance and usability when used in an Object Oriented fashion. Ruby with Ruby of Rails provides a very stable platform to develop upon, but does not quite have the flexibility that I intended to yield in the development. The final decision came between Python and PHP. Python is a clean language with very good performance and is easy to use. However, the flexibility that PHP provides when developing web pages and its good Object Oriented capabilities meant that I sided with it even though its performance may not be as good as Python.
		
	\subsection{JQuery}
		JQuery is featured on the website as a means of displaying certain features. Graphical maps generated by Google Directions \cite{google_directions_api} service are retrieved and displayed using JQuery. It is also used on the messages page of the site.	
			
	\subsection{PostgreSQL Database}
	Using an Object Relational database management system is the most efficient method of storing dynamic for websites. The PSQL database is handled by the object oriented PHP application that is used by the website.
	
	\subsection{Github}
		Github \cite{github} version control web hosting was chosen as the desired.
		 
\section{Overall Architecture}
	\subsection{Overview}	
		The data storage system the website uses is a PostgreSQL database which contains the following tables:
		\begin{itemize}
		\item Person - Personal details of each user.
		\item Journey - Details of Journeys that users have posted.
		\item Journey\textunderscore Step - Each journey has many journey steps. This table holds the geographical location, the related journey and the order of the step.
		\item Journey\textunderscore Step\textunderscore Temp - A temporary table used when the journey steps for a particular journey change. This may occur is a hitch request is accepted which alters the route of the journey.
		\item Hitch\textunderscore Request - Details about a hitch request made from a person to a particular journey.
		\item Message - Messages sent from user to user.
		\end{itemize}
		
		All actions performed on the database are done via a connection from the PHP database controller classes. Each table has a representative PHP model class mirroring the table. Controller classes control these classes to insert, update and delete records from each of the tables. 
		
		The website instantiates the database controller classes to enable the site to produce dynamic output and allow the user to access all of the site's features once logged in.
	\subsection{Method}
		The waterfall software development methodology was used as the development methodology for this project. A detailed project specification document was produced to outline the key functional requirements of the final release. 
		
		A design specification document was produced to outline the key design aspects of the site, including the database and database controller classes in PHP. It also outlined how the website would communicate with these controller classes.
		
		As well as the design specification document, UML design diagrams describing the database structure and database controller structure were produced. Use case diagrams describe how users interact with the website.
	\subsection{Structure of Database Controller}
		The Database Controller is the name given to the object oriented structure of PHP classes used to control data flow to and from the database. For each of the main tables in the database, there is a corresponding PHP class with the same name: Hitch\textunderscore Request, Journey, Journey\textunderscore Step, Message and Person. These classes that model the tables have attributes matching those of the table attributes and contain the 4 methods: Create, Load, Update, Delete; following the CRUD persistent storage technique\cite{crud_technique}. The Create method in each of the classes uses the class attribute values to create a new entry in the corresponding database table. The Update methods update the related entry already in the corresponding table using the attributes in the class object by the primary key that will have been retrieved from the database upon executing the Create method and stored as once of the class attributes. The Load method loads the attributes to the class object from the corresponding table in the database using the primary key as a parameter. The Delete method simply deletes the related entry from the corresponding table in the database using its primary key stored in the class object's attributes. The Delete method also resets all of the attribute values in the object just in case the object is used again for a different entry in the database. 
		
		Upon instantiating each of the model class objects, a non-compulsory parameter may be parsed to the constructor as the primary key to that table. If something is parsed in this parameter, the constructor will call the Load method, which attempts to populate the class object's attributes with values from the corresponding table using the parameter as the primary key.
		
		Each of the table model classes are utilized by controller classes. These classes instantiate their model classes to manipulate the entries in the database. Each of the controllers are unique to the tasks that need to be performed on each of the tables and often involve interaction between the controllers. For example, when a new Journey gets posted by a user, the Journey\textunderscore Controller class would be instantiated. This object would then instantiate a Journey class object, populate its attributes with values and use its Create method to insert the data into the database. It would then instantiate a Journey\textunderscore Step\textunderscore Controller class which in turn would instantiate and populate a series of Journey\textunderscore Step classes depending on the number of steps in that journey. Each of the entries would then be inserted into the database by the controller calling the Create method in each of the Journey\textunderscore Step objects. 		
		
\section{Website Design}
	\subsection{Overview}
		The website is produced by the collection of 25 PHP files which dynamically output HTML depending on the data that is received from the database controller classes. They also provide a platform for users to access all of the features available to them that the site offers.
	\subsection{PHP}
		The PHP files responsible for dynamically outputting the website maintain a session throughout the user's time on the site. This session allows the site to maintain a log on account for the user as they navigate through the site. The only data that is stored continuously as a session variable is the user's email address, which is used when they log in or register to the site.
		
	\subsection{Bootstrap}
	\subsection{JavaScript and JQuery Library}
	Google maps embedding

%\section{Overall Architecture}

%\section{Some detailed design}

%\subsection{Even more detail}

%\section{User Interface}

%\section{Other relevant sections}